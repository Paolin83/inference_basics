\documentclass[]{article}
\usepackage{lmodern}
\usepackage{amssymb,amsmath}
\usepackage{ifxetex,ifluatex}
\usepackage{fixltx2e} % provides \textsubscript
\ifnum 0\ifxetex 1\fi\ifluatex 1\fi=0 % if pdftex
  \usepackage[T1]{fontenc}
  \usepackage[utf8]{inputenc}
\else % if luatex or xelatex
  \ifxetex
    \usepackage{mathspec}
  \else
    \usepackage{fontspec}
  \fi
  \defaultfontfeatures{Ligatures=TeX,Scale=MatchLowercase}
\fi
% use upquote if available, for straight quotes in verbatim environments
\IfFileExists{upquote.sty}{\usepackage{upquote}}{}
% use microtype if available
\IfFileExists{microtype.sty}{%
\usepackage{microtype}
\UseMicrotypeSet[protrusion]{basicmath} % disable protrusion for tt fonts
}{}
\usepackage[margin=1in]{geometry}
\usepackage{hyperref}
\hypersetup{unicode=true,
            pdftitle={Manova},
            pdfauthor={livio finos},
            pdfborder={0 0 0},
            breaklinks=true}
\urlstyle{same}  % don't use monospace font for urls
\usepackage{graphicx,grffile}
\makeatletter
\def\maxwidth{\ifdim\Gin@nat@width>\linewidth\linewidth\else\Gin@nat@width\fi}
\def\maxheight{\ifdim\Gin@nat@height>\textheight\textheight\else\Gin@nat@height\fi}
\makeatother
% Scale images if necessary, so that they will not overflow the page
% margins by default, and it is still possible to overwrite the defaults
% using explicit options in \includegraphics[width, height, ...]{}
\setkeys{Gin}{width=\maxwidth,height=\maxheight,keepaspectratio}
\IfFileExists{parskip.sty}{%
\usepackage{parskip}
}{% else
\setlength{\parindent}{0pt}
\setlength{\parskip}{6pt plus 2pt minus 1pt}
}
\setlength{\emergencystretch}{3em}  % prevent overfull lines
\providecommand{\tightlist}{%
  \setlength{\itemsep}{0pt}\setlength{\parskip}{0pt}}
\setcounter{secnumdepth}{0}
% Redefines (sub)paragraphs to behave more like sections
\ifx\paragraph\undefined\else
\let\oldparagraph\paragraph
\renewcommand{\paragraph}[1]{\oldparagraph{#1}\mbox{}}
\fi
\ifx\subparagraph\undefined\else
\let\oldsubparagraph\subparagraph
\renewcommand{\subparagraph}[1]{\oldsubparagraph{#1}\mbox{}}
\fi

%%% Use protect on footnotes to avoid problems with footnotes in titles
\let\rmarkdownfootnote\footnote%
\def\footnote{\protect\rmarkdownfootnote}

%%% Change title format to be more compact
\usepackage{titling}

% Create subtitle command for use in maketitle
\providecommand{\subtitle}[1]{
  \posttitle{
    \begin{center}\large#1\end{center}
    }
}

\setlength{\droptitle}{-2em}

  \title{Manova}
    \pretitle{\vspace{\droptitle}\centering\huge}
  \posttitle{\par}
    \author{livio finos}
    \preauthor{\centering\large\emph}
  \postauthor{\par}
      \predate{\centering\large\emph}
  \postdate{\par}
    \date{10 novembre 2019}


\begin{document}
\maketitle

\section{Regressione Multivariata (MANOVA)}\label{MANOVA}

Le soluzioni ai problemi di verifica di ipotesi per uno e due campioni
risultano essere casi particolari di modelli più ampi. In analogia al
modo in cui i t-test (univariato) per uno e due campioni possono essere
formalizzati tramite modello di regressione (univariato), anche i
modelli esposti sopra risultano essere un caso particolare dei modelli
di regressione multivariati. Questi verranno introdotti qui di seguito.

\subsection{Definizione del modello e Stima dei parametri}

Il modello lineare\\
\[\YY_{n\times p}=\XX_{n\times q}\bfB_{q\times p} + \bfE_{n\times p}\]
con \(\bfE\) matrice normale di dati con righe indipendenti e
\(\epsilon_i\sim N(0,\Sigma)\ \forall i=1,\ldots,n\). Si noti che questo
equivale a dire che \(\YY\) è una m.n.d. da
\(y_i\sim N(\bfB' x_i,\Sigma)\).

La stima della matrice dei parametri \(\bfB\) e di \(\Sigma\) è
realizzata in stretta analogia con il modello univariato. Sia
\(\HH =\I - \XX(\XX'\XX)^{-1}\XX'=\I - \PP\) (vedi anche esercizio
\ref{Hcentramento}), la soluzione ai minimi quadrati è:

\[\hat{\bfB}=(\XX'\XX)^{-1}\XX'\YY\]
\[\hat{\Sigma}=S_u=(n-q)^{-1}\YY'\HH\YY\] e, di conseguenza,
\[\hat{\YY}=\XX{\hat \bfB} =\XX(\XX'\XX)^{-1}\XX'\YY=\PP\YY\] e
\textbackslash{} \[\hat{\bfE}=\YY-\hat{\YY}=\HH\YY=(\I-\PP)\YY.\]

\begin{ExerciseList}
\Exercise  \label{Hcentramento} Dimostrare che la definizione $\HH=I_n-\frac{1}{n} \bfuno\bfuno'$ già usata nel capitolo \ref{matriciDati} è un caso particolare di $\HH=\I - \XX(\XX'\XX)^{-1}\XX'$ (e quindi definire $\XX$ che verifica l'equivalenza).
\end{ExerciseList}

\subsection{Stimatori di massima verosimiglianza}

Per derivare più facilmente le proprietà inferenziali di questi
stimatori ai minimi quadrati è utile metterli in relazione con i
corrispettivi stimatori di massima verosimiglianza (SMV). Fortunatamente
i due metodi portano a risultati analoghi. Lo stimatore \(\hat{\bfB}\) è
lo stesso a cui si perviene tramite approccio dei minimi quadrati mentre
\(\hat{\Sigma}=S=(n)^{-1}\YY'\HH\YY\). Per derivare questi risultati, è
prima utile dimostrare il seguente:

\begin{lemma}\label{decoScarti}
Siano $y_i',\ i=1,\ldots,n$ i vettori riga di una matrice $\YY$, $x_i',\ i=1,\ldots,n$ i vettori riga di una matrice $\XX$  e ${\hat y}_i=\hat{\bfB}'x_i$. Fissate una qualsiasi matrice $\bfB_*$ e una matrice $\Sigma$ (definita positiva), vale
\begin{eqnarray*}
\sum_{i=1}^n (y_i-\bfB_*' x_i)'\Sigma^{-1}(y_i-\bfB_*' x_i)=\\
\tr\left( (\YY-X\bfB_*)\Sigma^{-1}(\YY-\XX\bfB_*)' \right)=\\
\tr \left( (\YY-\hat{\YY})\Sigma^{-1}(\YY-\hat{\YY})' \right) + \tr\left( (\hat{\YY}-\XX\bfB_*)\Sigma^{-1}(\hat{\YY}-\XX\bfB_*)' 
\right)
\end{eqnarray*}
\end{lemma}
\begin{proof}
La prima equazione è vera per definizione.
Per la seconda uguaglianza si ricordi la proprietà $\tr(A+B)=\tr(A)+\tr(B)$ per $A$ e $B$ matrici quadrate. Ora vale
\begin{eqnarray*}
\tr\left((\YY-\XX\bfB_*)\Sigma^{-1}(\YY-\XX\bfB_*)'\right) =\\
\tr\left((\YY-\hat{\YY}+\hat{\YY}-\XX\bfB_*)\Sigma^{-1}(\YY-\hat{\YY}+\hat{\YY}-\XX\bfB_*)' \right)=\\
\tr\left((\YY-\hat{\YY})\Sigma^{-1}(\YY-\hat{\YY})'\right) +\tr\left((\YY-\hat{\YY})\Sigma^{-1}(\hat{\YY}-\XX\bfB_*)'\right)+\\
\tr\left((\hat{\YY}-\XX\bfB_*)\Sigma^{-1}(\YY-\hat{\YY})'\right) + \tr\left((\hat{\YY}-\XX\bfB_*)\Sigma^{-1}(\hat{\YY}-\XX\bfB_*)'\right)
\end{eqnarray*}
Il risultato si ottiene se si osserva che il secondo e terzo termine sono nulli. Infatti, 
ricordando che vale $\tr(AB)=\tr(BA)$ per $A$ e $B$ matrici qualsiasi, per il secondo termine:
\begin{eqnarray*}
\tr\left((\YY-\hat{\YY})\Sigma^{-1}(\hat{\YY}-X\bfB_*)'\right)=\tr\left(\HH \YY\Sigma^{-1}(\PP\YY-X\bfB_*)'\right)=\\
\tr\left(\YY\Sigma^{-1}(\YY'\PP'-\bfB_*'\XX')\HH \right)=\tr\left(\YY\Sigma^{-1}(\YY'\PP'\HH-\bfB_*'\XX'\HH) \right)=\zerobold
\end{eqnarray*}
siccome $\YY' \PP'\HH = \YY \zerobold=\zerobold$ e $\bfB_*'\XX'\HH = \bfB_*'(\XX' - \XX'\XX (\XX'\XX)^{-1} \XX')=\bfB_* \zerobold$.
La stessa conclusione vale per il terzo termine.
\end{proof}

Ora possiamo mostrare il risultato più rilevante:

\begin{theorem}[Stima di massima verosimiglianza]
$\hat{\bfB}=(\XX'\XX)^{-1}\XX'\YY$ e $\hat{\Sigma}=S=\YY'\HH\YY/n$ sono gli stimatori di massima verosimiglianza di $\bfB$ e $\Sigma$, rispettivamente.
\end{theorem}
\begin{proof}
Omettendo il termine $2\pi^{p/2}$ che non ha alcun ruolo nella massimizzazione della log-verosimiglianza,
questa può essere esplicitata nel seguente modo:
\begin{eqnarray}\nonumber
l_{èl}(\bfB,\Sigma) &\propto& {-\frac{n}{2}} \log|\Sigma|     -\frac{1}{2} \sum_{i=1}^n (y_i-\bfB' x_i)'\Sigma^{-1}(y_i-\bfB' x_i)\\ \nonumber
&=& {-\frac{n}{2}} \log|\Sigma|     -\frac{1}{2} \tr\left( (\YY-\XX\bfB)\Sigma^{-1}(\YY-\XX\bfB)' \right)\\  \nonumber
&=& {-\frac{n}{2}} \log|\Sigma|     -\frac{1}{2} \tr\left( (\YY-\hat{\YY})\Sigma^{-1}(\YY-\hat{\YY})' \right) \\ 
&&-\frac{1}{2} \tr\left( (\hat{\YY}-\XX\bfB)\Sigma^{-1}(\hat{\YY}-\XX\bfB)' \right) \label{logLik}
\end{eqnarray}

Per la stima di $\bfB$ si noti che la matrice di parametri $\bfB$ è stata isolata nel terzo termine; questa quantità è sempre positiva e
massimizza la verosimiglianza quando si riduce a 0, cioè quando $\bfB = \hat{\bfB}$ (infatti $\hat{\YY}=\XX\hat{\bfB}$).

Per derivare lo stimatore di $\Sigma$, si osservi che per il secondo termine vale:
\begin{eqnarray}\label{SigmaS} \nonumber
-\frac{1}{2}\tr\left((\YY-\hat{\YY})\Sigma^{-1}(\YY-\hat{\YY})'\right)&=&-\frac{1}{2}\tr\left(\Sigma^{-1}(\YY-\hat{\YY})'(\YY-\hat{\YY})\right)=\\ 
&=&-\frac{n}{2}\tr(\Sigma^{-1} S)
\end{eqnarray}
(si ricordi che $\tr(AB)=\tr(BA)$).

Quindi 
$$l_{èl}(\bfB,\Sigma)\propto -\log|\Sigma|-\tr(\Sigma^{-1} S).$$
Mostriamo ora che il massimo è raggiunto in $\Sigma=S$.
Si definisca $A=\Sigma^{-1} S$ e si usi la proprietà $|AB|=|A||B|$ (ponendo $B=\Sigma$ si ha $|\Sigma|= |S|/|\Sigma^{-1}S|$) per riscrivere quest'ultima come 
$$l_{èl}(\bfB,\Sigma)\propto -\log|\Sigma|-\tr(\Sigma^{-1} S)=-\log|S|+\log|A| -\tr(A).$$
$\log|S|$ è una costante e può essere esclusa dalle considerazioni che seguono.

Definiti ora $\lambda_1,\ldots,\lambda_p$ gli autovalori di $A$ e ricordando che $|A|=\prod_i \lambda_i$ e $\tr(A)=\sum_i \lambda_i$, il problema
si riduce alla minimizzazione di $\sum_i^p (\lambda_i-\log \lambda_i)$. La funzione $f(\lambda)=\lambda-\log \lambda$ è minimizzata in $\lambda=1$ 
(ponendo la derivata pari a 0). Se tutti gli autovalori sono unitari $A=\I$ e quindi $\Sigma=S$.
\end{proof}

Questa dimostrazione ci permette di dire che gli stimatori
\(\hat{\bfB}\) e \(S\) sono asintoticamente efficienti (raggiungono il
limite inferiore di Rao-Cramer), però non garantiscono la stessa
proprietà nel finito, specie per campioni piccoli. Anche la non
distorsione è garantita asintoticamente; per \(\hat{\bfB}\) questa
risulta valida anche al finito, mentre per ricavare uno stimatore non
distorto di \(\Sigma\) è necessario fare ricorso allo stimatore ai
minimi quadrati \(S_u=Sn/(n-q)\).

Ciononostante, l'approccio di massima verosimiglianza resta fondamentale
per ricavare una forma generale del test basato sul rapporto di
verosimiglianza che vedremo tra poco.

\subsubsection*{Proprietà}

Anche le proprietà del modello estendono le proprietà del modello
lineare univariato.

\begin{theorem}
Dato il modello di regressione (lineare) e gli stimatori appena definiti:
\begin{itemize}
\item $\hat \bfB$ è uno stimatore non distorto di $\bfB$
\item $E(\hat{\bfE})=\zerobold$
\item $\hat{\bfB}$ (in vettore) ha distribuzione normale multivariata
\item $\hat{\bfB}\bbot \hat{\bfE}$
\item $nS \sim W_p(\Sigma, n-q)$ (e quindi uno stimatore non distorto è dato da $S_u=\frac{Sn}{n-q}$)
\end{itemize}
\end{theorem}

\subsection{Verifica di Ipotesi su $\bfB$}

Affronteremo ora il problema della verifica d'ipotesi su (sottomatrici
di) \(\bfB\). In molti casi faremo uso dell'approccio generale del test
di rapporto di log-verosimiglianza per ricavare alcuni tra i test già
visti nei capitoli precedenti.

`E quindi utile definirne la forma generale. Si supponga di voler
testare \(H_0:\ \bfB=\bfB_*\) contro \(H_1:\ \bfB\neq \bfB_*\)
(solitamente \(\bfB_*=0\)). Il rapporto di log-verosimiglianza è
definito come
\(\mathcal{W}=2 (l_{èl}(\hat{\bfB}_{|H_1})-l_{èl}(\hat{\bfB}_{|H_0})) = 2 (l_{1}-l_{0})\).

Si noti anche che sotto \(H_0\) e sotto condizioni di regolarità
asintoticamente vale \(\mathcal{W}\sim \chi^2_{pq}\).

\subsubsection{Test su $\mu$ con $\Sigma$ nota}

Sia \(\YY\) una m.n.d da \(y_i \sim N(\mu,\Sigma)\). Possiamo scrivere
questo modello fissando la matrice \(\XX=1_{n}\), cioè la media delle
righe di \(\YY\) viene modellizzata esclusivamente tramite una costante
e quindi \(E(\YY)=1_n\mu'=1_n\bfB\) (cioè \(\mu=\bfB'\))

Si voglia verificare l'ipotesi \(H_0: \mu=\mu_0=\bfB_0'\) contro
\(H_1: \mu\neq\mu_0=\bfB_0'\) con \(\mu_0\) fissato e \(\Sigma\) nota.
In questo caso, la log-verosimiglianza ha forma:
\[l_{èl}(\bfB,\Sigma) \propto 
-\frac{n}{2}\log|\Sigma|-\frac{1}{2}\tr\left((\YY -\bfB\XX)\Sigma^{-1}(\YY -\bfB\XX)'\right).\]

Sotto \(H_0\), possiamo sostituire \(\mu=\mu_0\); sotto \(H_1\) lo SMV
per \(1_n\hat{\mu}'=1_n(1_n'1_n)^{-1}1_n'\YY= 1_n\bar{y}\) e usando la
decomposizione (\ref{logLik}) e la equivalenza (\ref{SigmaS}), si
ottiene \begin{eqnarray*}
\mathcal{W}= l_{1}- l_{0} \propto 
&-&\frac{n}{2} \log|\Sigma|   -\frac{n}{2} \tr(\Sigma^{-1}S)  \\
&+&\frac{n}{2} \log|\Sigma|   +\frac{n}{2} \tr(\Sigma^{-1}S) 
+\frac{1}{2} \tr\left( (\hat{\YY}-1_n\mu_0')\Sigma^{-1}(\hat{\YY}-1_n\mu_0')' \right).
\end{eqnarray*} Si noti anche che
\(\mathcal{W}\propto \tr\left( (\hat{\YY}-1_n\mu_0')\Sigma^{-1}(\hat{\YY}-1_n\mu_0')' \right)=n (\hat{\mu}-\mu_0)'\Sigma^{-1}(\hat{\mu}-\mu_0)\)
Tutto si gioca quindi sulla distanza (di Mahalanobis) tra \(\hat{\mu}\)
e il vero (sotto \(H_0\)) \(\mu_0\). Quindi possiamo fare inferenza
usando tale distanza; sappiamo già dal paragrafo \ref{testSigmaNota} che
questo è un modo valido per fare inferenza, ora ne ricaviamo anche che è
un buon modo.

\subsubsection{Test su $\mu$ con $\Sigma$ ignota}\label{oneSample}

Il modello e le ipotesi sono le stesse del paragrafo precedente, però
non assumiamo \(\Sigma\) nota. Riprendiamo la forma compatta della
(log)verosimiglianza come definita in (\ref{logLik}). Ricordando che
\(\hat{\bfB}_{|H_0}=\bfB_0'=\mu_0\) e
\(\hat{\bfB}_{|H_1}=\hat{\bfB}'=\bar{y}\) se ne deduce che
\[\hat{\Sigma}_{|H_1}=(\YY-1\bar{y}')'(\YY-1\bar{y}')/n=S\] e
\begin{eqnarray*}
\hat{\Sigma}_{|H_0}&=&(\YY-1\mu_0)'(\YY-1\mu_0)/n\\
&=&(\YY-1\bar{y}')'(\YY-1\bar{y}')/n +(1\bar{y}'-1\mu_0')'(1\bar{y}'-1\mu_0')/n\\
&=&S+(\bar{y}-\mu_0)(\bar{y}-\mu_0)'.
 \end{eqnarray*} Sempre con riferimento a (\ref{logLik}), si nota che il
secondo termine si riduce a \(-np/2\) sia sotto \(H_0\), sia sotto
\(H_1\) (infatti sotto \(H_1\):
\(\tr\left( (\YY-1\mu')\hat{\Sigma}^{-1}(\YY-1\mu')' \right)= \tr\left( \hat{\Sigma}^{-1}(\YY-1\mu')'(\YY-1\mu') \right)=n\tr(I_p)\)
e analogamente sotto \(H_0\)). Quindi il rapporto delle due massime
log-verosimiglianze risulta funzione esclusiva degli stimatori delle
varianze sotto le due ipotesi:
\[\mathcal{W}\propto n\log\frac{|S_{H_0}|}{|S_{H_1}|}=n\log\frac{|S+(\bar{y}-\mu_0)(\bar{y}-\mu_0)'|}{|S|}.\]
Ora sfrutteremo due proprietà dei determinati :

\begin{itemize}
\item[i ] $|AB|/|B|=|A|$ (ottenuta da $|AB|=|A||B|$) e 
\item[ii ] $|I_p+BC|=|I_n+CB|$ (con $B_{p\times n}$ e $C_{n\times p}$, usata con $p=1$).
\end{itemize}

Possiamo notare che
\(|S+(\bar{y}-\mu_0)(\bar{y}-\mu_0)'| =|S(I+S^{-1}(\bar{y}-\mu_0)(\bar{y}-\mu_0)')|\)
e che
\(|(I+S^{-1}(\bar{y}-\mu_0)(\bar{y}-\mu_0)'|=|(1+(\bar{y}-\mu_0)'S^{-1}(\bar{y}-\mu_0)|\).
Ne consegue che
\(\mathcal{W}=n\log(|AB|/|B|)=n\log(|A|)=n\log(|1 + (\bar{y}-\mu_0)'S^{-1}(\bar{y}-\mu_0)|)\)
dove, essendo l'argomento uno scalare, il determinante può essere
omesso. \((\bar{y}-\mu_0)'S^{-1}(\bar{y}-\mu_0)\) è esattamente
l'espressione della \(T^2\) a meno del fattore \(n-1\) e quindi ne è
funzione monotona.

Questo dimostra che il test \(T^2\) già ricavato tramite il lemma
\ref{T2media}, è anche un test del rapporto di verosimiglianza.

\subsubsection{Test su $\bfB$ in presenza di covariate}

'E molto comune che la verifica di ipotesi riguardi solo una parte dei
parametri e cioè delle righe di \(\bfB\) (si noti che se solo alcune
colonne sono di interesse per il test, sarà sufficiente escludere le
altre variabili/colonne dal modello). Il caso più tipico è quello in cui
si vuole vedere la relazione tra le variabili \(\XX\) e \(\YY\)
ammettendo però che l'intercetta sia non nulla.

In generale decomporremo la matrice \(\XX\) in due sottomatrici
\(\XX_1\) e \(\XX_0\) (di rango rispettivamente \(q_1\) e \(q_0\)),
facenti riferimento rispettivamente ai parametri sotto test \(\bfB_1\) e
a quelli che possono assumere valori non nulli anche sotto l'ipotesi
nulla \(\bfB_0\). \begin{eqnarray}\label{full}
\mathbf{\YY} = \XX_0\bfB_0 +  \XX_1\bfB_1 + \mathbf{E}
\end{eqnarray}

L'ipotesi nulla da testare risulta quindi
\[H_0: \bfB_1=\zerobold \ \forall \bfB_0\]

\bigskip

Tale approccio si rivela estremamente flessibile. Nel caso di
\(c\geq 2\) due o più campioni indipendenti è sufficiente definire
\(\XX_0\) come la matrice con una sola colonna di uni (l'intercetta) e
\(\XX_1\) come matrice di \(c-1\) variabili dummy (indicatrici di
gruppo, esclusa la categoria di riferimento). Tale caso particolare è ha
volte detto modello MANOVA.

Il caso di 2 (o \(c>2\)) campioni appaiati può essere incluso in questo
schema definendo in \(\XX_0\) le dummy ricavate dagli indicatori dei
soggetti e in \(\XX_1\) l'indicatore dei due (o \(c\)) trattamenti. Si
noti, inoltre, che è possibile includere il caso di un singolo campione
definendo \(\XX_0=\emptyset\) e \(\XX_1=1_n\). Più in generale si
possono includere disegni sperimentali a blocchi ed anche più complessi
in questo schema generale.

\begin{ExerciseList}
\Exercise Definire la matrice $\XX$ di un modello di regressione multivariato che formalizzi il confronto di due campioni di numerosità $n_1$ ed $n_2$ (le prime $n_1$ osservazioni di $\YY$ provengono dal primo campione).
\Exercise Formalizzare $\XX_0$ e $\XX_1$ per la verifica dell'ipotesi nulla di uguaglianza dei due campioni.
\end{ExerciseList}
\bigskip

A partire da questa impostazione, è possibile decomporre la devianza
totale in modo del tutto analogo a quello che accade per i modelli
univariati. Sia \(\HH_0=\I - \XX_0(\XX_0'\XX_0)^{-1}\XX_0'\), si
definiscano le seguenti quantità:

\begin{itemize}
\item Devianza Totale (sotto $H_0$): $T=\YY'\HH_0\YY$
\item Devianza Residua (sotto $H_1$):  $W=\YY'\HH\YY$
\item Devianza Spiegata (sotto $H_1$) da $\XX_1$: $B=\YY'\HH_0\YY-\YY'\HH\YY=\YY'(\HH_0-\HH)\YY$
\end{itemize}

Si noti che grazie al teorema \ref{CraigLancaster} di Craig - Lancaster
possiamo ricavare che le tre matrici casuali sono indipendenti tra loro,
seguono una distribuione di Wishart e la loro somma è una whishart con
g.d.l. pari al rango di \(\HH_0\). In ambito univariato, la statistica
test F è basata sul rapporto
\(F=\frac{B/rango(\HH_0-\HH)}{W/rango(\HH)}\) la cui distrbuzione è
nota. Vale anche la pena notare che lo stesso indice di proporzione di
varianza spiegata \(R^2=\frac{B}{T}=\frac{B}{B+W}\).

In ambito multivariato le stastistiche di sintesi non sono altrettanto
immediate. Nei casi non banali, la statistica test non può basarsi sul
rapporto delle due devianze giacch'e numeratore e denominatore sono
delle matrici. `E necessario quindi ``sintetizzare'' queste due matrici
in un'unica quantità.

Si noti che un caso particolare si verifica quando \(\XX_1\) ha una sola
colonna (e quindi \(rango(\HH_0-\HH)=1\)). Nel ricavare la distribuzione
della statistica test del rapporto di verosimiglianze (paragrafo
\ref{oneSample}) abbiamo sfruttato la proprietà ii. per scoprire che
questa segue una \(T^2\). In generale questo risulta vero non solo per
il caso di due campioni, ma più in generale nel caso di un solo
predittore sotto test.

Continuando sulla strada segnata del test del rapporto di
verosimiglianza, possiamo ricavare il test per il generico problema di
regressione. In modo del tutto analogo a quando mostrato in
\ref{oneSample} (e facendo ancora riferimento ai risultati del lemma
\ref{decoScarti}), possiamo notare che il secondo termine in
\ref{logLik} si riduce a \(-np/2\) sia sotto \(H_0\) che \(H_1\)
(infatti sotto \(H_1\):
\(\tr\left( (\YY-\XX\hat{\bfB})\hat{\Sigma}^{-1}(\YY-\XX\hat{\bfB})' \right)= \tr\left( \hat{\Sigma}^{-1}(\YY-\XX\hat{\bfB})'(\YY-\XX\hat{\bfB}) \right)=n\tr(I_p)\)
e analogamente sotto \(H_0\) usando \(\XX_0\) e \(\bfB_0\)). Il test del
rapporto di verosimiglianza quindi si riduce ad una funzione monotona
dei determinanti delle varianze (funzione delle devianze) residue sotto
\(H_0\) ed \(H_1\):
\[\mathcal{W}\propto n\log\frac{|S_{H_0}|}{|S_{H_1}|}=n\log\frac{|T|/n}{|W|/n}=n\log\frac{|W+B|}{|W|}.\]

Come anticipato per`o, la proprietà ii. usata nel paragrafo
\ref{oneSample} non ci permette di giungere a nessuna forma distributiva
nota. Siamo quindi costretti a sfruttare i risultati asintotici
(\(\mathcal{W}\) ha distribuzione asintotica \(\chi^2_{pq_1}\) --
\(q_1\) e \(p\) dimensioni di \(\bfB\))\\
oppure ricavare la forma della nuova variabile aleatoria.

\subsubsection{$\Lambda$ di Wilks e altre misure di sintesi}
\subsubsection*{$\Lambda$ di Wilks}

il rapporto delle verosimiglianze
\[\lambda^{2/n} \propto |W| /|T| =  |W| |W+B|^{-1} =  |\I +W^{-1}B|^{-1} \sim \Lambda(p,n-q_0-q_1,q_1) \]
(parametri: \(p=\)colonne Y,\(n-q_0-q_1=\)rango di W, \(q_1=\)rango di
\(B\))

Ora, sotto \(H_0\), \(X\) `e una matrice di dati da
\(\N_p(\bfmu,\Sigma)\), cos`i dal teorema di Cochran generalizzato si
ha, ponendo \(C_1=\HH\) e \(C_2=\HH_0-\HH\) \[
W=X'C_1X \sim W_p(\Sigma,n-q_0-q_1)
\] \[
B=X'C_2X \sim W_p(\Sigma,q_1)
\] Dal momento che \(B\) e \(W\) sono indipendenti, a condizione che
\(n-q_0-q_1> 0\), sotto \(H_0\) la statistica \(\lambda^{2/n}\) si
distribuisce come una \(\Lambda\) di Wilks \[
|I+W^{-1}B|^{-1} \sim \Lambda(p,n-q_0-q_1,q_1)
\] Purtroppo solo in casi particolari - cio`e per particolari valori dei
parametri- si pu`o utilmente sfruttare la distribuzione al finito di
\(\Lambda\).

\begin{center}\rule{0.5\linewidth}{\linethickness}\end{center}

\begin{theorem}
Si ha
$$
\Lambda(p,m,n) \sim \prod_{i=1}^nu_i
$$
dove $u_1,\ldots,u_n$ sono $n$ variabili indipendenti e $u_i \sim \beta(\frac{1}{2}
(m+i-p), \frac{1}{2}p), i=1,\ldots,n$.
\end{theorem}

\{\em Senza dimostrazione\}.

Per valori grandi di \(m\) si pu`o anche utilizzarsi il risultato
asintotico.\textbackslash{}

\begin{theorem} 
$$
-[m-\frac{1}{2}(p-n+1)]\log\Lambda(p,m,n) \sim \chi^2_{np}
$$
per $m \rightarrow \infty$.
\end{theorem}

\{\em Senza dimostrazione\}.

\subsubsection*{Altre misure di sintesi della varianza spiegata}

Come detto, la statistica di sintesi \(\lambda=|W|/|T|\) non è l'unica
possibile. Va anche sottolineato che in ambito multivariato, non è
facile stabilire criteri di ottimalità; ad esempio non esite un test
uniformemente più potente. La scelta di \(\lambda\) quindi è
giustificabile per l'importanza che ricopre l'approccio di
verosimiglianza in tutto l'ambito inferenziale e per il fatto che sia
nota la sua distribuzione asintotica, ma non gode di migliori proprietà
rispetto alle altre stastistiche di sintesi che indicheremo qui di
seguito.

Partendo dalla quantità \(A=W^{-1}B\) è possibile stabilire diverse
misure di sintesi. Siano \(\lambda_1, \ldots,\lambda_p\) gli autovalori
non nulli di \(A=W^{-1}B\) Tra le molte, le più note sono:

\begin{itemize}
\item Wilks: $\prod_i (1/(1+\lambda_i))\sim \Lambda$ come già visto
\item (Traccia di) Pillai: $\sum_i (1/(1+\lambda_i))$
\item (Traccia di) Lawley-Hotelling: $\sum_i \lambda_i$
\item (Massima radice di) Roy: $\max_i \lambda_i$
\end{itemize}

Nessuna di queste statistiche ha una distribuzione nota (o facile da
ricavare) sotto l'ipotesi nulla. I progessi tecnologici degli ultimi
decenni, però ci permettono di ottenere queste distribuzioni tramite
simulazione.


\end{document}
